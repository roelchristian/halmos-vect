\addchap{Recommended Reading}

The following very short listmakes no pretense to completeness; it merely
contains a couple of representatives of each of several directions in which the
reader may want to proceed.

\bigskip

For generalized (but usually finite-dimensional) linearand multilinear algebra:
\begin{enumerate}[nosep]
    \item N. {\scshape Bourbaki}, \emph{Algèbre}; Chap. II (\emph{Algèbre linéaire}), Paris, 1947, and Chap. III (\emph{Algèbre multilinéaire}), Paris, 1948.
    \item B. L. {\scshape van der Waerden}, \emph{Modern algebra}, New York, 1953.
\end{enumerate}

\medskip

For connections with classical and modern analysis:
\begin{enumerate}[nosep]
    \item S. {\scshape Banach}, \emph{Théorie des opérations linéaires}, Warszawa,1932.
    \item F. {\scshape Rumsz} and B. {\scshape Sz.-Nagy}, \emph{Functional analysis}, New York, 1955.
\end{enumerate}

\medskip

For the geometry of Hilbert space and transformations on it:
\begin{enumerate}[nosep]
    \item P. R. {\scshape Halmos}, \emph{Introduction to Hilbert space}, New York, 1951.
    \item M. H. {\scshape Stone}, \emph{Linear transformations in Hilbert space}, New York, 1932.
\end{enumerate}

\medskip

For contact with classical and modern physics:
\begin{enumerate}[nosep]
    \item R. {\scshape Courant} and D. {\scshape Hilbert}, \emph{Methods of mathematical physics}, New York, 1953. 
    \item J. {\scshape von Neumann}, \emph{Mathematical foundations of quantum mechanics}, Princeton, 1955.
\end{enumerate}