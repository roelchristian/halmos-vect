
\addchap{Preface}

My purpose in this book is to treat linear transformations on finite-dimensional
vector spaces by the methods of more general theories. The idea is to emphasize
the simple geometric notions common to many parts of mathematics and its
applications,and to do so in a language that gives away the trade secrets and
tells the student what is in the back of the minds of people proving theorems
about integral equations and Hilbert spaces. The reader does not, however, have
to share my prejudiced motivation. Except for an occasional reference to
undergraduate mathematics the book is self-contained and may be read by anyone
who is trying to get a feeling for the linear problems usually discussed in
courses on matrix theory or ``higher'' algebra. The algebraic, coordinate-free
methods do not lose power and elegance by specialization to a finite number of
dimensions, and they are, in my belief, as elementary as the classical
coordinatized treatment.

I originally intended this book to contain a theorem if and only if an
infinite-dimensional generalization of it already exists. The tempting easiness
of some essentially finite dimensional notions and results was, however,
irresistible, and in the final result my initial intentions are just barely
visible. They are most clearly seen in the emphasis, throughout, on
generalizable methods instead of sharpest possible results. The reader may
sometimes see some obvious way of shortening the proofs I give. In such cases
the chances are that the infinite dimensional analogue of the shorter proof is
either much longer or else non-existent.

A preliminary edition of the book (Annals of Mathematics Studies, Number 7,
first published by the Princeton University Press in 1942) has been circulating
for several years. In addition to some minor changes in style and in order, the
difference between the preceding version and this one is that the latter
contains the following new material: (1) A brief discussion of fields, and, in
the treatment of vector spaces with inner products, special attention to the
real case. (2) A definition of determinants in invariant terms, via the theory
of multilinear forms. (3) Exercises.


The exercises (well over three hundred of them) constitute the most significant
addition; I hope that they will be found useful by both student and teacher.
There are two things about them the reader should know. First, if an exercise is
neither imperative (``prove that \,.\,.\,.\,'') nor interrogative (``is it true
that \,.\,.\,.\, ?'') but merely declarative, then it is intended as a
challenge. For such exercises the reader is asked to discover if the assertion
is true or false, prove it if true and construct a counterexample if false, and,
most important of all, discuss such alterations of hypothesis and conclusion as
will make the true ones false and the false ones true. Second, the exercises,
whatever their grammatical form, are not always placed so as to make their very
position a hint to their solution. Frequently exercises are stated as soon as
the statement makes sense, quite a bit before machinery for a quick solution has
been developed. A reader who tries (even unsuccessfully) to solve such a
``misplaced'' exercise is likely to appreciate and to understand the subsequent
developments much better for his attempt. Having in mind possible future
editions of the book, I ask the reader to let me know about errors in the
exercises, and to suggest improvements and additions. (Needless to say, the same
goes for the text.)

None of the theorems and only very few of the exercises are my discovery; most
of them are known to most working mathematicians, and have been known for along
time. Although I do not give a detailed list of my sources, I am nevertheless
deeply aware of my indebtedness to the books and papers from which I learned and
to the friends and strangers who, before and after the publication of the first
version, gave me much valuable encouragement and criticism. I am particularly
grateful to three men: J. L. Doob and Arlen Brown, who read the entire
manuscript of the first and the second version, respectively, and made many
useful suggestions, and John von Neumann, who was one of the originators of the
modern spirit and methods that I have tried to present and whose teaching was
the inspiration for this book.

\begin{flushright}
    P. R. H.
\end{flushright}